% !TEX encoding = UTF-8
\documentclass[a4paper,oneside,titlepage]{book}
%\usepackage[T1]{fontenc}
\usepackage[italian]{babel}

\usepackage{amsmath}
\usepackage{booktabs}
\usepackage[nouppercase]{frontespizio}
\usepackage{graphicx}
\graphicspath{{./immagini/}}
\usepackage{hyperref}
\usepackage{listings}
\usepackage{xcolor}

% Impostazioni per i listati di codice:
\definecolor{codegray}{rgb}{0.5,0.5,0.5}
\definecolor{backcolour}{rgb}{0.95,0.95,0.92}
\lstdefinestyle{mystyle}{
    backgroundcolor=\color{backcolour},
    numberstyle=\tiny\color{codegray},
    basicstyle=\ttfamily\footnotesize,
    breakatwhitespace=false,         
    breaklines=true,                 
    captionpos=b,                    
    keepspaces=true,                 
    numbers=left,                    
    numbersep=5pt,                  
    showspaces=false,                
    showstringspaces=false,
    showtabs=false,                  
    tabsize=2
}
\lstset{style=mystyle}

% Comandi:
\newcommand{\xlongleftrightarrow}[1]{\overset{#1}{\longleftrightarrow}}
\newcommand{\mail}[1]{\href{mailto:#1}{\texttt{#1}}}


\begin{document}

\begin{frontespizio}
	\Universita{Verona}
	\Dipartimento{Informatica}
	\Corso[Laurea]{Informatica}
	\Titolo{Guida agli esercizi per il corso di Sistemi}
	\NCandidato{Creato da}
	\Candidato{Davide Zampieri}
	\Annoaccademico{2018-2019}
\end{frontespizio}

\frontmatter
\tableofcontents

\mainmatter
\chapter{Sistemi LTI a tempo continuo}

\section{Segnali elementari}
\begin{itemize}
\item Impulso
\[ \delta (t) = \begin{cases}
1 & se \,\, t = 0 \\
0 & altrimenti
\end{cases} \]

\item Gradino
\[ \delta_{-1} (t) = \begin{cases}
1 & se \,\, t \geq 0 \\
0 & se \,\, t < 0
\end{cases} \]
\end{itemize}

\textbf{Nota:}
ricorda che $\delta_{-1} (t) = \int \delta(t)\,dt$ e che $\frac{d \delta(t)}{dt} = \delta_{-1}(t)$


\section{Forma di un sistema LTI continuo}
\begin{itemize}
\item Forma generale
\[ \sum_{i=0}^n a_i \frac{d^{i} v(t)}{dt^{i}} = \sum_{j=0}^m b_j \frac{d^{j} u(t)}{dt^{j}} \]

\item Forma usata negli esercizi
\[ a_2 \ddot{v}(t) + a_1 \dot{v}(t) + a_0 v(t) = b_2 \ddot{u}(t) + b_1 \dot{u}(t) + b_0 u(t) \]
\end{itemize}


\section{Analisi nel tempo}
La risposta totale del sistema si scrive come
\[ v(t) = v_l(t) + v_f(t) \]
dove $v_l(t)$ è la risposta libera e $v_f(t)$ è la risposta forzata.

\subsection{Risposta libera}
Per trovare la risposta libera bisogna:
\begin{enumerate}
\item Trovare le radici $\lambda_{1,2}$ dell'equazione caratteristica delle uscite
\[ a_2 s^2 + a_1 s + a_0 = 0 \]

\item Scrivere la risposta libera come
\[ v_l(t) = \sum_{i=1}^r \sum_{l=0}^{\mu_i-1} c_{i,l} \cdot e^{\lambda_it} \cdot \frac{t^l}{l!} \]

\item Usare le condizioni iniziali sulle uscite

\textbf{Esempio:}
se le molteplicità $\mu_i$ delle radici sono tutte pari a 1, avremo
\[ v_l(t) = c_1 \cdot e^{\lambda_1t} + c_2 \cdot e^{\lambda_2t} \]
\[ \dot{v}_l(t) = c_1 \cdot \lambda_1 \cdot e^{\lambda_1t} + c_2 \cdot \lambda_2 \cdot e^{\lambda_2t} \]
e quindi bisognerà risolvere il sistema
\[ \begin{cases}
c_1 + c_2 = v(0) \\
c_1 \cdot \lambda_1 + c_2 \cdot \lambda_2 = \dot{v}(0)
\end{cases} \]
\end{enumerate}

\subsection{Stabilità asintotica}
Grazie ai modi elementari, ovvero le radici $\lambda_i$ dell'equazione caratteristica delle uscite, possiamo studiare la stabilità asintotica di un sistema LTI.

Infatti, un sistema LTI a tempo continuo è asintoticamente stabile se e solo se $Re(\lambda_i)<0$.

Infine, si può dire che: se un sistema LTI è asintoticamente stabile, allora è anche BIBO stabile.

\subsection{Risposta impulsiva}
Per trovare la risposta impulsiva bisogna:
\begin{enumerate}
\item Scrivere la forma generale
\[ h(t) = d_0 \cdot \delta(t) + \sum_{i=1}^r \sum_{l=0}^{\mu_i-1} d_{i,l} \cdot e^{\lambda_it} \cdot \frac{t^l}{l!} \cdot \delta_{-1}(t) \]

\textbf{Nota:}
il termine con il coefficiente $d_0$ è presente solo quando il sistema LTI ha $n = m$

\item Porre $v(t) = h(t)$ e $u(t) = \delta(t)$ per trovare i coefficienti $d_i$

\textbf{Esempio:}
se il sistema LTI ha $n=2$, bisognerà calcolare la derivata prima e la derivata seconda della risposta impulsiva ricordando che i termini che moltiplicano il gradino vanno eliminati; a questo punto basta raccogliere i termini che moltiplicano l'impulso e le sue derivate e risolvere un sistema
\end{enumerate}

\subsection{Risposta forzata}
La risposta forzata si può calcolare come
\[ v_f(t) = [h*u](t) = \int_{0}^t h(\tau) \cdot u(t-\tau)\,d\tau \]
che è equivalente a
\[ v_f(t) = [u*h](t) = \int_{0}^t u(\tau) \cdot h(t-\tau)\,d\tau \]


\section{Analisi nelle frequenze}
A volte lavorare nel dominio delle frequenze è più semplice che lavorare in quello del tempo.

\subsection{Trasformata di Laplace}
Per passare dal dominio del tempo a quello delle frequenze si utilizza la trasformata di Laplace. Di seguito alcune trasformate notevoli:
\begin{itemize}
\item Uscite
\[ \mathcal{L} \left[ \frac{d^{i} v(t)}{dt^{i}} \right] = s^i \cdot V(s) - \left( \sum_{k=0}^i \left. \frac{d^{k} v(t)}{dt^{k}} \right|_{t=0^-} \cdot s^{i-1-k} \right) \]

\item Ingressi
\[ \mathcal{L} \left[ \frac{d^{i} u(t)}{dt^{i}} \right] = s^i \cdot U(s) \]

\item Esponenziale causale
\[ \mathcal{L} \left[ e^{\lambda t} \cdot \delta_{-1}(t) \right] = \frac{1}{s - \lambda} \]
\end{itemize}

\subsection{Risposta totale}
Una volta calcolate le trasformate di Laplace di tutti i termini del sistema LTI si arriva ad una forma del tipo
\[ V(s) = V_l(s) + V_f(s) = V_l(s) + H(s) \cdot U(s) = \frac{P(s)}{D(s)} + \frac{N(s)}{D(s)} \cdot U(s) \]
dove $D(s)$ è l'equazione caratteristica delle uscite, $N(s)$ è l'equazione caratteristica degli ingressi e $P(s)$ è la parte relativa alle condizioni iniziali sulle uscite.

\subsection{BIBO stabilità}
Per studiare la BIBO stabilità è necessario ricavare la funzione di trasferimento $H(s)$, che abbiamo visto essere pari a $\frac{N(s)}{D(s)}$.

In particolare, bisogna vedere se i poli $p_i$ (ovvero le radici di $D(s)$) rimasti dopo eventuali semplificazioni con gli zeri $z_i$ (ovvero le radici di $N(s)$) hanno la parte reale negativa. Quindi, un sistema LTI a tempo continuo è BIBO stabile se e solo se $Re(p_i)<0$.

\subsection{Metodo dei fratti semplici}
Partendo dalla forma usata negli esercizi si arriva, usando la trasformata di Laplace, ad una forma del tipo
\[ a_2 [s^2 \cdot V(s) - (v(0) \cdot s + \dot{v}(0))] + a_1 [s \cdot V(s) - v(0)] + a_0 \cdot V(s) = \]
\[ = b_2 s^2 \cdot U(s) + b_1 s \cdot U(s) + b_0 \cdot U(s) \]
A questo punto, dopo alcuni calcoli, si può riuscire a trovare la risposta totale del sistema usando il metodo dei fratti semplici. Di seguito alcuni esempi:
\begin{itemize}
\item Tre poli di molteplicità pari a 1
\[ V(s) = \frac{f(s)}{(s-\lambda_1)(s-\lambda_2)(s-\lambda_3)} = \frac{A}{s-\lambda_1} + \frac{B}{s-\lambda_2} + \frac{C}{s-\lambda_3} \]
\[ A = \left. (s-\lambda_1) \cdot \frac{f(s)}{(s-\lambda_1)(s-\lambda_2)(s-\lambda_3)} \right|_{s=\lambda_1} \]
\[ B = \left. (s-\lambda_2) \cdot \frac{f(s)}{(s-\lambda_1)(s-\lambda_2)(s-\lambda_3)} \right|_{s=\lambda_2} \]
\[ C = \left. (s-\lambda_3) \cdot \frac{f(s)}{(s-\lambda_1)(s-\lambda_2)(s-\lambda_3)} \right|_{s=\lambda_3} \]

\item Due poli di cui uno di molteplicità pari a 2
\[ V(s) = \frac{f(s)}{(s-\lambda_1)(s-\lambda_2)^2} = \frac{A}{s-\lambda_1} + \frac{B}{s-\lambda_2} + \frac{C}{(s-\lambda_2)^2} \]
\[ A = \left. (s-\lambda_1) \cdot \frac{f(s)}{(s-\lambda_1)(s-\lambda_2)^2} \right|_{s=\lambda_1} \]
\[ B = \frac{d}{ds} \left. \left( (s-\lambda_2)^2 \cdot \frac{f(s)}{(s-\lambda_1)(s-\lambda_2)^2} \right) \right|_{s=\lambda_2} \]
\[ C = \left. (s-\lambda_2)^2 \cdot \frac{f(s)}{(s-\lambda_1)(s-\lambda_2)^2} \right|_{s=\lambda_2} \]
\end{itemize}
\textbf{Nota:}
il metodo dei fratti semplici si può usare anche per trovare la sola risposta libera, forzata o impulsiva. In quest'ultimo caso però, occorre ricordare che se il sistema LTI ha $n=m=2$, la forma da cui partire diventa
\[ H(s) = b_2 + \frac{A}{s-\lambda_1} + \frac{B}{s-\lambda_2} \]

\subsection{Anti-trasformata di Laplace}
Una volta trovata la soluzione nel dominio delle frequenze, è possibile tornare nel dominio del tempo utilizzando l'anti-trasformata di Laplace. Di seguito alcune anti-trasformate notevoli:
\begin{itemize}
\item Costante
\[ \mathcal{L}^{-1} \left[ A \right] = A \cdot \delta(t) \]

\item Fratti semplici
\[ \mathcal{L}^{-1} \left[ \frac{A}{(s - \lambda)^\alpha} \right] = A e^{\lambda t} \cdot \frac{t^{\alpha -1}}{(\alpha -1)!} \cdot \delta_{-1}(t) \]
\end{itemize}


\section{Studio della stabilità al variare di k}
Se in un sistema LTI è presente un parametro $k$, la condizione per garantire la stabilità asintotica (ovvero $Re(\lambda_i)<0$) diventa
\[ \begin{cases}
+ \frac{c}{a} > 0 \\
- \frac{b}{a} < 0
\end{cases} \]
dove $a$, $b$ e $c$ sono i coefficienti delle uscite.

Anche in questo caso possiamo dire che: se un sistema LTI è asintoticamente stabile, allora è anche BIBO stabile.


\chapter{Schemi a blocchi e diagrammi di flusso}

\section{Trasmittanza totale}
Dopo aver trasformato uno schema a blocchi in un diagramma di flusso, è possibile trovare la funzione di trasferimento tra ingresso e uscita andando a calcolare la trasmittanza totale come
\[ T = \frac{\sum_i P_i \cdot \Delta_i}{\Delta} \]
con
\[ \Delta = 1 - \sum_j P_{j1} + \sum_j P_{j2} \]
dove $P_i$ sono i cammini aperti, $P_{j1}$ sono gli anelli singoli, $P_{j2}$ sono le coppie di anelli che non si toccano e $\Delta_i$ sono uguali a $\Delta$ ma senza i contributi degli anelli che toccano il cammino aperto $P_i$ a cui si riferiscono.

\section{Esempi di esercizi con soluzione in Matlab}
\subsection{Esempio 1}
Si vuole calcolare la funzione di trasferimento del seguente schema a blocchi
\begin{figure}[htp]
    \centering
    \includegraphics[scale=.5]{blocchi1.PNG}
    \caption{Esempio 1}
    \label{fig:blocchi1}
\end{figure}

\begin{lstlisting}[language=Matlab, caption=Esempio 1 in Matlab]
syms G1 G2 G3
syms u y e z h w
sys = [e==u-h; z==G1*e; y==G2*z; h==z+w; w==G3*y];
sol = solve(sys,[e,z,y,h,w]);
Gtot = coeffs(sol.y,u)
\end{lstlisting}
Con le seguenti relazioni ingresso-uscita
\begin{itemize}
\item $e = u-h$
\item $z = G1 \cdot e$
\item $y = G2 \cdot z$
\item $h = z + w$
\item $w = G3 \cdot y$
\end{itemize}
la funzione di trasferimento trovata è
\[ Gtot = \frac{G1 \cdot G2}{G1 + G1 \cdot G2 \cdot G3 + 1} \]

\subsection{Esempio 2}
Si vuole calcolare la funzione di trasferimento del seguente schema a blocchi
\begin{figure}[htp]
    \centering
    \includegraphics[scale=.5]{blocchi2.PNG}
    \caption{Esempio 2}
    \label{fig:blocchi2}
\end{figure}

\begin{lstlisting}[language=Matlab, caption=Esempio 2 in Matlab]
syms G1 G2 H1 H2
syms u y e1 e2 y1 y2
sys = [e1==u-y-y2; e2==G1*e1-y1; y1==H2*y; y2==H1*e2; y==G2*e2];
sol = solve(sys,[e1,e2,y1,y2,y]);
Gtot = coeffs(sol.y,u)
\end{lstlisting}
Con le seguenti relazioni ingresso-uscita
\begin{itemize}
\item $e_1 = u - y - y_2$
\item $e_2 = G1 \cdot e_1 - y_1$
\item $y_1 = H2 \cdot y$
\item $y_2 = H1 \cdot e_2$
\item $y = G2 \cdot e_2$
\end{itemize}
la funzione di trasferimento trovata è
\[ Gtot = \frac{G1 \cdot G2}{G1 \cdot G2 + G1 \cdot H1 + G2 \cdot H2 + 1} \]

\subsection{Esempio 3}
Si vuole calcolare la funzione di trasferimento del seguente schema a blocchi
\begin{figure}[htp]
    \centering
    \includegraphics[scale=.5]{blocchi3.PNG}
    \caption{Esempio 3}
    \label{fig:blocchi3}
\end{figure}

\begin{lstlisting}[language=Matlab, caption=Esempio 3 in Matlab]
syms A B C D E F
syms u y e1 e2 y1 y2 y3
sys = [e1==u-C*y1; e2==y1+y2-y3; y1==B*e1; y2==A*u; y3==D*E*e2; y==(E+F)*e2];
sol = solve(sys,[e1,e2,y1,y2,y3,y]);
Gtot = coeffs(sol.y,u)
\end{lstlisting}
Con le seguenti relazioni ingresso-uscita
\begin{itemize}
\item $e_1 = u - C \cdot y_1$
\item $e_2 = y_1 + y_2 - y_3$
\item $y_1 = B \cdot e_1$
\item $y_2 = A \cdot u$
\item $y_3 = D \cdot E \cdot e_2$
\item $y = (E+F) \cdot e_2$
\end{itemize}
la funzione di trasferimento trovata è
\[ Gtot = \frac{(E + F) \cdot (A + B + A \cdot B \cdot C)}{(B \cdot C + 1) \cdot (D \cdot E + 1)} \]


\chapter{Trasformata di Fourier}

\section{Trasformate notevoli}
Dato uno schema a blocchi, è possibile trovare l’uscita del sistema per via grafica lavorando nel dominio delle frequenze, utilizzando la trasformata di Fourier e le sue proprietà. Di seguito alcune trasformate notevoli:
\begin{itemize}
\item Funzione costante
\[ \mathcal{F} \left[ A \right] = A \cdot \delta(f) \]

\item Fasore
\[ \mathcal{F} \left[ A \cdot e^{j 2\pi f_0 t} \right] = A \cdot \delta(f-f_0) \]

\item Funzione coseno
\[ \mathcal{F} \left[ A \cdot \cos(2\pi f_0 t) \right] = \frac{A}{2} \left( \delta(f-f_0) + \delta(f+f_0) \right) \]

\item Funzione seno
\[ \mathcal{F} \left[ A \cdot \sin(2\pi f_0 t) \right] = \frac{A}{2j} \left( \delta(f-f_0) - \delta(f+f_0) \right) \]

\item Finesta rettangolare
\[ \mathcal{F} \left[ A \cdot \Pi \left( \frac{t}{T} \right) \right] = AT \cdot sinc(fT) \]
\end{itemize}

\textbf{Nota:} valgono anche in senso inverso

\section{Proprietà}
\begin{itemize}
\item \textbf{Prodotto nel dominio del tempo:} in via grafica consiste nel centrare il segnale $v_1(t)$ nel segnale $v_2(t)$ o viceversa.
\[ v_1(t) \cdot v_2(t) \xlongleftrightarrow{\mathcal{F}} V_1(f) * V_2(f) \]

\item \textbf{Convoluzione nel dominio del tempo:} in via grafica consiste nell'applicare il filtro $v_2(t)$ al segnale $v_1(t)$; bisogna cioè tenere le parti di $v_1(t)$ contenute in $v_2(t)$.
\[ v_1(t) * v_2(t) \xlongleftrightarrow{\mathcal{F}} V_1(f) \cdot V_2(f) \]

\item \textbf{Campionamento nel dominio del tempo:} in via grafica consiste nel replicare il segnale $v(t)$ ogni $f = \frac{1}{T}$; bisogna cioè centrare $v(t)$ ogni $f$ Hz.
\[ [samp_T \, v](t) \xlongleftrightarrow{\mathcal{F}} \frac{1}{T} [rep_{\frac{1}{T}} \, V](f) \]
\end{itemize}

\section{Ampiezze dei segnali}
Nelle operazioni di prodotto e convoluzione, le ampiezze dei due segnali vanno sempre moltiplicate tra di loro.

Le ampiezze vanno invece sommate solo nel caso in cui due segnali si sovrappongano.

Nell'operazione di campionamento, infine, tutti i segnali vanno moltiplicati per la frequenza di campionamento $f = \frac{1}{T}$.

\section{Aliasing}
Il fenomeno di aliasing si verifica quando $f < 2B$, dove con $B$ si intende la banda del segnale, ovvero quanto il segnale è largo nelle frequenze positive.

L'aliasing consiste nella possibile sovrapposizione di due o più segnali, dovuta al fatto che si va a replicare secondo una frequenza che è minore dello spazio che occupa l'intero segnale.


\chapter{Diagrammi di Bode}

\section{Forma di Bode}
Data una funzione di trasferimento nella forma
\[ G(s) = K \cdot \frac{\prod_i (s-z_i)^{\mu_i} \cdot \prod_k (s-z_k)(s- \overline{z_k})}{\prod_i (s-p_i)^{\mu_i} \cdot \prod_k (s-p_k)(s- \overline{p_k})} \]
è possibile passare alla sua forma di Bode definita come
\[ G(j\omega) = K_B \cdot \prod_l \frac{1}{(j\omega)^{\nu_l}} \cdot \prod_i (1+j\omega \tau_i)^{\mu_i} \cdot \prod_k \left( 1+2j\zeta_k \cdot \frac{\omega}{\omega_{n_k}} - \frac{\omega^2}{\omega^{\,2}_{n_k}} \right)^{\mu_k} \]

\section{Diagrammi elementari}
Di seguito si elencano le formule per ricavare i diagrammi di modulo e fase di ogni termine della forma di Bode.

\noindent
\begin{tabular}{cc}
    \toprule
    NOME e FORMA & MODULO e FASE  \\
    \midrule
    \textbf{Termine costante:} & $|H(j\omega)|_{dB} = 20 \cdot log|K_B|$ \\
    $H(j\omega)=K_B$ & $arg(H(j\omega)) =
		\begin{cases}
		0 & se \,\, K_B > 0 \\
		-180 & se \,\, K_B < 0
		\end{cases}$ \\
	\midrule
	\textbf{Zeri e poli nell'origine:} & $|H(j\omega)|_{dB} = 20 \cdot (-\nu) \cdot log|\omega|$ \\
	$H(j\omega)=\frac{1}{(j\omega)^\nu}$ & $arg(H(j\omega)) = -\nu \cdot 90$ \\
	\midrule
	\textbf{Zeri e poli reali:} & $|H(j\omega)|_{dB} =
		\begin{cases}
		0 & se \,\, \omega \ll \frac{1}{|\tau|} \\
		20 \mu \cdot log|\omega\tau| & se \,\, \omega \gg \frac{1}{|\tau|}
		\end{cases}$ \\
	$H(j\omega)=(1+j\omega\tau)^\mu$ & $arg(H(j\omega)) =
		\begin{cases}
		0 & se \,\, \omega \ll \frac{1}{|\tau|} \\
		90 \mu \cdot sgn(\tau) & se \,\, \omega \gg \frac{1}{|\tau|}
		\end{cases}$ \\
	\midrule
	\textbf{Zeri e poli complessi:} & $|H(j\omega)|_{dB} =
		\begin{cases}
		0 & se \,\, \omega \ll \omega_n \\
		40 \mu \cdot log \left| \frac{\omega}{\omega_n} \right| & se \,\, \omega \gg \omega_n
		\end{cases}$ \\
	$H(j\omega) = \left( 1+2j\zeta \frac{\omega}{\omega_n} - \frac{\omega^2}{\omega^{\, 2}_n} \right)^\mu$ & $arg(H(j\omega)) =
		\begin{cases}
		0 & se \,\, \omega \ll \omega_n \\
		180 \mu \cdot sgn(\zeta) & se \,\, \omega \gg \omega_n
		\end{cases}$ \\
	\bottomrule
\end{tabular}

\subsection{Note}
\begin{itemize}
\item Il diagramma del modulo per zeri e poli nell'origine passa per $10^0$

\item Il diagramma della fase per zeri e poli reali si può approssimare facendolo passare per i seguenti punti
\[ A = \left( \frac{1}{5|\tau|}, 0 \right) \]
\[ B = \left( \frac{5}{|\tau|}, 90 \mu \cdot sgn(\tau) \right) \]

\item Il diagramma della fase per zeri e poli complessi si può approssimare facendolo passare per i seguenti punti
\[ A = \left( \frac{1}{5^{|\zeta|}} \cdot \omega_n, 0 \right) \]
\[ B = \left( 5^{|\zeta|} \cdot \omega_n, 180 \mu \cdot sgn(\zeta) \right) \]
\end{itemize}

\section{Esempio di esercizio (convenzione Matlab)}
Si vuole rappresentare la seguente funzione di trasferimento
\[ H(s) = \frac{s-2}{(s+4) \cdot (s+1)} \]
che scritta in forma di Bode diventa
\[ H(s) = - \frac{1}{2} \cdot \frac{\left( 1 - \frac{1}{2} s \right)}{\left( 1 + \frac{1}{4} s \right) \cdot (1+s)} \]
Non resta che rappresentare i quattro fattori elementari trovati e sommare i vari contributi:
\begin{itemize}
\item $H1(s) = - \frac{1}{2}$
\item $H2(s) = 1 - \frac{1}{2} s$
\item $H3(s) = \left( 1 + \frac{1}{4} s \right)^{-1}$
\item $H4(s) = (1+s)^{-1}$
\end{itemize}
\newpage
\begin{lstlisting}[language=Matlab, caption=Codice Matlab]
H1 = tf([-0.5],[1]);
H2 = tf([-0.5 1],[1]);
H3 = tf([1],[0.25 1]);
H4 = tf([1],[1 1]);
H = tf([1 -2],[1 5 4]);

subplot(2,3,1)
bode(H1,'r')
title('H1(s)')
subplot(2,3,2)
bode(H2,'g')
title('H2(s)')
subplot(2,3,3)
bode(H3,'b')
title('H3(s)')
subplot(2,3,4)
bode(H4,'m')
title('H4(s)')
subplot(2,3,5)
bode(H1,'r',H2,'g',H3,'b',H4,'m',H,'k')
title('Contributi di H(s)')
subplot(2,3,6)
bode(H,'k')
title('Somma dei contributi di H(s)')
\end{lstlisting}

\begin{figure}[htp]
    \centering
    \includegraphics[width=\textwidth, height=\textheight, keepaspectratio]{bode.PNG}
    \caption{Diagrammi di Bode}
    \label{fig:bode}
\end{figure}


\chapter{Sistemi LTI a tempo discreto}

\section{Forma di un sistema LTI discreto}
\begin{itemize}
\item Forma generale
\[ \sum_{i=0}^n a_i \cdot v(k-i) = \sum_{j=0}^m b_j \cdot u(k-j) \]

\item Forma usata negli esercizi $(n \geq m)$
\[ a_0 v(k) + a_1 v(k-1) + a_2 v(k-2) = b_0 u(k) + b_1 u(k-1) + b_2 u(k-2) \]
\end{itemize}


\section{Analisi nel tempo}
La risposta totale del sistema si scrive come
\[ v(k) = v_l(k) + v_f(k) \]
dove $v_l(k)$ è la risposta libera e $v_f(k)$ è la risposta forzata.

\subsection{Risposta libera}
Per trovare la risposta libera bisogna:
\begin{enumerate}
\item Trovare le radici $\lambda_{1,2}$ dell'equazione caratteristica delle uscite
\[ \left. a_0 z^0 + a_1 z^{-1} + a_2 z^{-2} = 0 \right|_{\cdot z^n = z^2} \]
\[ a_0 z^2 + a_1 z + a_2 = 0 \]

\item Scrivere la risposta libera come
\[ v_l(k) = \sum_{i=1}^r \sum_{l=0}^{\mu_i-1} c_{i,l} \cdot \lambda^{\,k}_i \cdot \frac{k^l}{l!} \]

\item Usare le condizioni iniziali sulle uscite

\textbf{Esempio:}
se le molteplicità $\mu_i$ delle radici sono tutte pari a 1, avremo
\[ v_l(k) = c_1 \cdot \lambda^{\,k}_1 + c_2 \cdot \lambda^{\,k}_2 \]
e quindi bisognerà risolvere il sistema
\[ \begin{cases}
c_1 \cdot \lambda^{\,-1}_1 + c_2 \cdot \lambda^{\,-1}_2 = v(-1) \\
c_1 \cdot \lambda^{\,-2}_1 + c_2 \cdot \lambda^{\,-2}_2 = v(-2)
\end{cases} \]
\end{enumerate}

\subsection{Stabilità asintotica}
Grazie ai modi elementari, ovvero le radici $\lambda_i$ dell'equazione caratteristica delle uscite, possiamo studiare la stabilità asintotica di un sistema LTI.

Infatti, un sistema LTI a tempo discreto è asintoticamente stabile se e solo se $|\lambda_i|<1$.

Infine, si può dire che: se un sistema LTI è asintoticamente stabile, allora è anche BIBO stabile.

\subsection{Risposta impulsiva}
Per trovare la risposta impulsiva bisogna:
\begin{enumerate}
\item Scrivere la forma generale
\[ h(k) = d_0 \cdot \delta(k) + \sum_{i=1}^r \sum_{l=0}^{\mu_i-1} d_{i,l} \cdot \lambda^{\,k}_i \cdot \frac{k^l}{l!} \cdot \delta_{-1}(k-m+n-1) \]
\textbf{Nota:}
il termine con il coefficiente $d_0$ è presente solo quando il sistema LTI ha $n = m$

\item Trovare i coefficienti $d_i$ ponendo $v(k) = h(k)$, $u(k) = \delta(k)$ e cercando una funzione ricorsiva andando a sostituire nell'equazione ottenuta diversi valori di $k$ fino a che non si arriva a riconoscere uno schema ricorsivo; dopodiché basta sostituire i valori della funzione ricorsiva appena trovata nella forma generale e risolvere un sistema
\end{enumerate}

\subsection{Risposta forzata}
La risposta forzata si può calcolare come
\[ v_f(k) = [h*u](k) = \sum_{i=0}^k h(i) \cdot u(k-i) \]
che è equivalente a
\[ v_f(k) = [u*h](k) = \sum_{i=0}^k u(i) \cdot h(k-i) \]
La risposta forzata si può calcolare anche come funzione ricorsiva andando a sostituire nel sistema LTI diversi valori di $k$ fino a che non si arriva a riconoscere uno schema ricorsivo.


\section{Analisi nelle frequenze}
A volte lavorare nel dominio delle frequenze è più semplice che lavorare in quello del tempo.

\subsection{Trasformata zeta}
Per passare dal dominio del tempo a quello delle frequenze (e viceversa) si utilizza la trasformata zeta. Di seguito alcune trasformate notevoli:
\begin{itemize}
\item Uscite
\[ \mathcal{Z} \left[ v(k-i) \right] = z^{-i} \cdot V(z) + \left( \sum_{l=-i}^{-1} v(l) \cdot z^{-i-l} \right) \]

\item Ingressi
\[ \mathcal{Z} \left[ u(k-i) \right] = z^{-i} \cdot U(z) \]

\item Impulso
\[ \mathcal{Z} \left[ \delta(k) \right] = 1 \]

\item Gradino
\[ \mathcal{Z} \left[ \delta_{-1}(k) \right] = \frac{z}{z - 1} \]

\item Modo elementare di molteplicità pari a 1
\[ \mathcal{Z} \left[ \lambda^k \cdot \delta_{-1}(k) \right] = \frac{z}{z - \lambda} \]

\item Modo elementare di molteplicità pari a 2
\[ \mathcal{Z} \left[ k \cdot \lambda^k \cdot \delta_{-1}(k) \right] = \frac{z \lambda}{(z - \lambda)^2} \]
\end{itemize}

\subsection{Risposta totale}
Una volta calcolate le trasformate zeta di tutti i termini del sistema LTI si arriva ad una forma del tipo
\[ V(z) = V_l(z) + V_f(z) = V_l(z) + H(z) \cdot U(z) = \frac{P(z)}{D(z)} + \frac{N(z)}{D(z)} \cdot U(z) \]
dove $D(z)$ è l'equazione caratteristica delle uscite, $N(z)$ è l'equazione caratteristica degli ingressi e $P(z)$ è la parte relativa alle condizioni iniziali sulle uscite.

\subsection{BIBO stabilità}
Per studiare la BIBO stabilità è necessario ricavare la funzione di trasferimento $H(z)$, che abbiamo visto essere pari a $\frac{N(z)}{D(z)}$.

In particolare, bisogna vedere se i poli $p_i$ (ovvero le radici di $D(z)$) rimasti dopo eventuali semplificazioni con gli zeri $z_i$ (ovvero le radici di $N(z)$) sono in modulo minori di 1. Quindi, un sistema LTI a tempo discreto è BIBO stabile se e solo se $|p_i|<1$.

\subsection{Metodo dei fratti semplici}
Partendo dalla forma usata negli esercizi si arriva, usando la trasformata zeta, ad una forma del tipo
\[ a_0 [z^0 \cdot V(z)] + a_1 [z^{-1} \cdot V(z) + v(-1) \cdot z^0] + a_2 [z^{-2} \cdot V(z) + v(-1) \cdot z^{-1} + v(-2) \cdot z^0] = \]
\[ \left. = b_0 z^0 \cdot U(z) + b_1 z^{-1} \cdot U(z) + b_2 z^{-2} \cdot U(z) \right|_{\cdot z^n = z^2} \]
\[ a_0 [z^2 \cdot V(z)] + a_1 [z \cdot V(z) + v(-1) \cdot z^2] + a_2 [V(z) + v(-1) \cdot z + v(-2) \cdot z^2] = \]
\[ = b_0 z^2 \cdot U(z) + b_1 z \cdot U(z) + b_2 \cdot U(z) \]
A questo punto, dopo alcuni calcoli, si può riuscire a trovare la risposta totale del sistema usando il metodo dei fratti semplici come per i sistemi LTI a tempo continuo, ricordando però di dividere per $z$ prima dei calcoli e di moltiplicare per $z$ dopo gli stessi, per riuscire poi a fare la trasformata zeta. La sequenza dei passaggi sarà quindi la seguente:
\begin{enumerate}
\item Trovo la risposta totale $V(z)$

\item Divido per $z$

\[ V_1(z) = \frac{V(z)}{z} \]

\item Uso il metodo dei fratti semplici

\item Moltiplico per $z$
\[ V(z) = V_1(z) \cdot z \]

\item Faccio la trasformata zeta e ottengo $v(k)$
\end{enumerate}
\textbf{Nota:}
il metodo dei fratti semplici si può usare anche per trovare la sola risposta libera, forzata o impulsiva.


\appendix
\chapter{Esempi di esercizi svolti}

\section{Sistemi LTI a tempo continuo}

\subsection{Esercizio 1}
\paragraph{Testo:}
\[ \ddot{v}(t) + \dot{v}(t)	- 2 v(t) = \dot{u}(t) + u(t) \]
\[ v(0) = 2, \, \dot{v}(0) = 0, \, u(t) = e^{-3t} \cdot \delta_{-1}(t) \]

\paragraph{Stabilità asintotica:}
\[ s^2 + s - 2 = 0 \]
\[ \lambda_{1,2} = \frac{-1 \pm \sqrt{1 + 8}}{2} = \frac{-1 \pm 3}{2}\]
\[ \lambda_1 = 1, \, \lambda_2 = -2 \]
Il sistema non è asintoticamente stabile perché $Re(\lambda_1)>0$

\paragraph{Risposta libera:}
\[ v_l(t) = c_1 \cdot e^t + c_2 \cdot e^{-2t} \]
\[ \dot{v}_l(t) = c_1 \cdot e^t - 2 c_2 \cdot e^{-2t} \]
\[
\begin{cases}
c_1 + c_2 = 2 \\
c_1 - 2 c_2 = 0
\end{cases}
\longrightarrow
\begin{cases}
c_1 = 2 - c_2 \\
2 - c_2 -2 c_2 = 0
\end{cases}
\longrightarrow
\begin{cases}
c_1 = \frac{4}{3} \\
c_2 = \frac{2}{3}
\end{cases}
\]
La risposta libera del sistema è $v_l(t) = \frac{4}{3} \cdot e^t + \frac{2}{3} \cdot e^{-2t}$

\paragraph{Risposta impulsiva:}
\[ h(t) = (d_1 \cdot e^t + d_2 \cdot e^{-2t}) \cdot \delta_{-1}(t) \]
\[
\dot{h}(t) = (d_1 \cdot e^t -2 d_2 \cdot e^{-2t}) \cdot \delta_{-1}(t)
+ (d_1 \cdot e^t + d_2 \cdot e^{-2t}) \cdot \delta(t)
\]
\[ \ddot{h}(t) = (d_1 \cdot e^t +4 d_2 \cdot e^{-2t}) \cdot \delta_{-1}(t)
+ (d_1 \cdot e^t -2 d_2 \cdot e^{-2t}) \cdot \delta(t) + \]
\[ + (d_1 \cdot e^t -2 d_2 \cdot e^{-2t}) \cdot \delta(t)
+ (d_1 \cdot e^t + d_2 \cdot e^{-2t}) \cdot \frac{d \delta(t)}{dt} \]
\[
\begin{cases}
v(t) = h(t) \\
u(t) = \delta(t)
\end{cases}
\longrightarrow
\ddot{h}(t) + \dot{h}(t) -2 h(t) = \frac{d \delta(t)}{dt} + \delta(t)
\]
\[
\delta(t) \cdot (3d_1 \cdot e^t -3 d_2 \cdot e^{-2t} -1) + \frac{d \delta(t)}{dt} \cdot (d_1 \cdot e^t + d_2 \cdot e^{-2t} -1) = 0
\]
\[
\begin{cases}
3d_1 -3d_2 - 1 = 0 \\
d_1 + d_2 - 1 = 0
\end{cases}
\longrightarrow
\begin{cases}
3 - 3d_2 - 3d_2 - 1 = 0 \\
d_1 = 1 - d_2
\end{cases}
\longrightarrow
\begin{cases}
d_2 = \frac{1}{3} \\
d_1 = \frac{2}{3}
\end{cases}
\]
La risposta impulsiva del sistema è $h(t) = \left( \frac{2}{3} \cdot e^t + \frac{1}{3} \cdot e^{-2t} \right) \cdot \delta_{-1}(t)$

\paragraph{Risposta forzata:}
\[ v_f(t) = [h*u](t) = \int_{0}^t h(\tau) \cdot u(t-\tau) \, d\tau = \]
\[
= \int_{0}^t \left( \frac{2}{3} \cdot e^\tau + \frac{1}{3} \cdot e^{-2\tau} \right) \cdot \delta_{-1}(\tau) \cdot e^{-3(t-\tau)} \cdot \delta_{-1}(t-\tau) \, d\tau =
\]
\[
= e^{-3t} \cdot \int_{0}^t \left( \frac{2}{3} \cdot e^{4\tau} + \frac{1}{3} \cdot e^\tau \right) \, d\tau = e^{-3t} \cdot \left( \frac{2}{3} \cdot \frac{1}{4} \cdot \int_{0}^t 4e^{4\tau} \, d\tau + \frac{1}{3} \cdot \int_{0}^t e^\tau \, d\tau \right) =
\]
\[
= e^{-3t} \cdot \left( \frac{1}{6} \cdot \left[ e^{4\tau} \right]_0^t + \frac{1}{3} \cdot \left[ e^\tau \right]_0^t \right) = e^{-3t} \cdot \left[ \frac{1}{6} \cdot (e^{4t} - 1) + \frac{1}{3} \cdot (e^t - 1) \right] =
\]
\[
= \frac{1}{6} e^{t} - \frac{1}{6} e^{-3t} + \frac{1}{3} e^{-2t} - \frac{1}{3} e^{-3t} = \frac{1}{6} e^{t} - \frac{1}{2} e^{-3t} + \frac{1}{3} e^{-2t}
\]

\paragraph{BIBO stabilità:}
\[ H(s) = \frac{s + 1}{s^2 + s - 2} = \frac{s + 1}{(s - 1)(s + 2)} \]
Il sistema non è BIBO stabile perché $Re(p_1)>0$

\paragraph{Risposta impulsiva (fratti semplici):}
\[ H(s) = \frac{A}{s - 1} + \frac{B}{s + 2} \]
\[ A = \left. (s-1) \cdot \frac{s + 1}{(s-1)(s+2)} \right|_{s=1} = \left. \frac{s + 1}{s + 2} \right|_{s=1} = \frac{2}{3} \]
\[ B = \left. (s+2) \cdot \frac{s + 1}{(s-1)(s+2)} \right|_{s=-2} = \left. \frac{s + 1}{s - 1} \right|_{s=-2} = \frac{1}{3} \]
\[
h(t) = \mathcal{L}^{-1} [H(s)] = \mathcal{L}^{-1} \left[ \frac{2}{3} \cdot \frac{1}{s - 1} + \frac{1}{3} \cdot \frac{1}{s + 2} \right] = \left( \frac{2}{3} \cdot e^{t} + \frac{1}{3} \cdot e^{-2t} \right) \cdot \delta_{-1}(t)
\]

\paragraph{Risposta forzata (fratti semplici):}
\[ U(s) = \mathcal{L} \left[ e^{-3t} \cdot \delta_{-1}(t) \right] = \frac{1}{s+3} \]
\[ V_f(s) = H(s) \cdot U(s) = \frac{s + 1}{(s - 1)(s + 2)(s + 3)} = \frac{A}{s - 1} + \frac{B}{s + 2} + \frac{C}{s + 3} \]
\[ A = \left. (s-1) \cdot \frac{s+1}{(s-1)(s+2)(s+3)} \right|_{s=1} = \left. \frac{s+1}{(s+2)(s+3)} \right|_{s=1} = \frac{1}{6} \]
\[ B = \left. (s+2) \cdot \frac{s+1}{(s-1)(s+2)(s+3)} \right|_{s=-2} = \left. \frac{s+1}{(s-1)(s+3)} \right|_{s=-2} = \frac{1}{3} \]
\[ C = \left. (s+3) \cdot \frac{s+1}{(s-1)(s+2)(s+3)} \right|_{s=-3} = \left. \frac{s+1}{(s-1)(s+2)} \right|_{s=-3} = - \frac{1}{2} \]
\[
v_f(t) = \mathcal{L}^{-1} [V_f(s)] = \mathcal{L}^{-1} \left[ \frac{1}{6} \cdot \frac{1}{s - 1} + \frac{1}{3} \cdot \frac{1}{s + 2} - \frac{1}{2} \cdot \frac{1}{s + 3} \right]
\]
La risposta forzata del sistema è $v_f(t) = \left( \frac{1}{6} \cdot e^{t} + \frac{1}{3} \cdot e^{-2t} - \frac{1}{2} \cdot e^{-3t} \right) \cdot \delta_{-1}(t)$


\subsection{Esercizio 2}
\paragraph{Testo:}
\[ \ddot{v}(t) - \dot{v}(t)	- 2 v(t) = \ddot{u}(t) + 2 \dot{u}(t) + u(t) \]
\[ v(0) = 1, \, \dot{v}(0) = -1, \, u(t) = e^{-3t} \cdot \delta_{-1}(t) \]

\paragraph{Stabilità asintotica:}
\[ s^2 - s - 2 = 0 \]
\[ \lambda_{1,2} = \frac{1 \pm \sqrt{1 + 8}}{2} = \frac{1 \pm 3}{2}\]
\[ \lambda_1 = 2, \, \lambda_2 = -1 \]
Il sistema non è asintoticamente stabile perché $Re(\lambda_1)>0$

\paragraph{Risposta libera:}
\[ v_l(t) = c_1 \cdot e^{2t} + c_2 \cdot e^{-t} \]
\[ \dot{v}_l(t) = 2c_1 \cdot e^{2t} - c_2 \cdot e^{-t} \]
\[
\begin{cases}
c_1 + c_2 = 1 \\
2c_1 - c_2 = -1
\end{cases}
\longrightarrow
\begin{cases}
c_1 = 1 - c_2 \\
2 - 2c_2 - c_2 = -1
\end{cases}
\longrightarrow
\begin{cases}
c_1 = 0 \\
c_2 = 1
\end{cases}
\]
La risposta libera del sistema è $v_l(t) = e^{-t}$

\paragraph{Risposta impulsiva (caso particolare $n=m$):}
\[ h(t) = d_0 \cdot \delta(t) + (d_1 \cdot e^{2t} + d_2 \cdot e^{-t}) \cdot \delta_{-1}(t) \]
\[
\dot{h}(t) = d_0 \cdot \frac{d \delta(t)}{dt} + (2d_1 \cdot e^{2t} - d_2 \cdot e^{-t}) \cdot \delta_{-1}(t) + (d_1 \cdot e^{2t} + d_2 \cdot e^{-t}) \cdot \delta(t)
\]
\[ \ddot{h}(t) = d_0 \cdot \frac{d^2 \delta(t)}{dt^2}
+ (4d_1 \cdot e^{2t} + d_2 \cdot e^{-t}) \cdot \delta_{-1}(t) +
+ 2 \cdot (2d_1 \cdot e^{2t} - d_2 \cdot e^{-t}) \cdot \delta(t) + \]
\[ + (d_1 \cdot e^{2t} + d_2 \cdot e^{-t}) \cdot \frac{d \delta(t)}{dt} \]
\[
\begin{cases}
v(t) = h(t) \\
u(t) = \delta(t)
\end{cases}
\longrightarrow
\ddot{h}(t) - \dot{h}(t) -2 h(t) = \frac{d^2 \delta(t)}{dt^2} + 2 \frac{d \delta(t)}{dt} + \delta(t)
\]
\[
\delta(t) \cdot (3d_1 e^{2t} - 3d_2 e^{-t} - 2d_0 - 1)
+ \frac{d \delta(t)}{dt} \cdot (d_1 e^{2t} + d_2 e^{-t} - d_0 - 2)
+ \frac{d^2 \delta(t)}{dt^2} \cdot (d_0 - 1) = 0
\]
\[
\begin{cases}
3d_1 - 3d_2 - 2d_0 - 1 = 0 \\
d_1 + d_2 - d_0 - 2 = 0 \\
d_0 - 1 = 0
\end{cases}
\longrightarrow
\begin{cases}
-3d_2 + 9 - 3d_2 - 3 = 0 \\
d_1 = -d_2 + 3 \\
d_0 = 1
\end{cases}
\longrightarrow
\begin{cases}
d_2 = 1 \\
d_1 = 2 \\
d_0 = 1
\end{cases}
\]
La risposta impulsiva del sistema è $h(t) = \delta(t) + (2 \cdot e^{2t} + e^{-t}) \cdot \delta_{-1}(t)$

\paragraph{BIBO stabilità:}
\[ H(s) = \frac{s^2 + 2s + 1}{s^2 - s - 2} = \frac{(s + 1)^2}{(s - 2)(s + 1)} = \frac{s+1}{s-2} \]
Il sistema non è BIBO stabile perché $Re(p_1)>0$

\paragraph{Risposta impulsiva (fratti semplici con caso particolare $n=m$):}
\[ H(s) = 1 + \frac{A}{s - 2} + \frac{B}{s + 1} \]
\[ A = \left. (s-2) \cdot \frac{(s + 1)^2}{(s - 2)(s + 1)} \right|_{s=2} = \left. \frac{(s + 1)^2}{s + 1} \right|_{s=2} = 3 \]
\[ B = \left. (s+1) \cdot \frac{(s + 1)^2}{(s - 2)(s + 1)} \right|_{s=-1} = \left. \frac{(s + 1)^2}{s - 2} \right|_{s=-1} = 0 \]
\[
h(t) = \mathcal{L}^{-1} [H(s)] = \mathcal{L}^{-1} \left[ 1 + \frac{3}{s - 2} + \frac{0}{s + 1} \right]
= \delta(t) + \left( 3 \cdot e^{2t} + 0 \cdot e^{-t} \right) \cdot \delta_{-1}(t)
\]

\paragraph{Nota:}
si può notare che la risposta impulsiva calcolata in questo modo è diversa da quella calcolata precedentemente, ciò è dovuto al fatto che l'impulso e le sue derivate sono linearmente indipendenti e quindi le soluzioni trovate sono combinazioni lineari 


\newpage
\subsection{Esercizio 3}
\paragraph{Testo:}
\[ \ddot{v}(t) - 2(k-1) \dot{v}(t) + (k+5) v(t) = 2 \dot{u}(t) - u(t) \]
\[ v(0) = 2, \, \dot{v}(0) = -3, \, u(t) = e^{-3t} \cdot \delta_{-1}(t) \]

\paragraph{Stabilità:}
\[
\begin{cases}
k+5 > 0 \\
2(k-1) < 0
\end{cases}
\longrightarrow
\begin{cases}
k > -5 \\
k < 1
\end{cases}
\longrightarrow
-5 < k < 1
\]
Se $-5 < k < 1$, il sistema è asintoticamente stabile e quindi anche BIBO stabile

\paragraph{Risposta totale (con $k=-1$):}
\[ U(s) = \mathcal{L} \left[ e^{-3t} \cdot \delta_{-1}(t) \right] = \frac{1}{s+3} \]
\[ s^2 \cdot V(s) - (2s - 3) + 4 \cdot (s \cdot V(s) - 2) + 4 \cdot V(s) = 2s \cdot U(s) - U(s) \]
\[ V(s) \cdot (s^2 + 4s + 4) - 2s + 3 - 8 = U(s) \cdot (2s - 1) \]
\[ V(s) = \frac{2s - 1}{s^2 + 4s + 4} \cdot U(s) + \frac{2s + 5}{s^2 + 4s + 4} = \frac{2s - 1}{(s+2)^2 \cdot (s+3)} + \frac{2s + 5}{(s+2)^2} = \]
\[
= \frac{2s-1 + (2s+5)(s+3)}{(s+2)^2 \cdot (s+3)}
= \frac{2s^2 + 13s + 14}{(s+2)^2 \cdot (s+3)}
= \frac{A}{s+3} + \frac{B}{s+2} + \frac{C}{(s+2)^2}
\]
\[ A = \left. (s+3) \cdot \frac{2s^2 + 13s + 14}{(s+2)^2 \cdot (s+3)} \right|_{s=-3} = \left. \frac{2s^2 + 13s + 14}{(s+2)^2} \right|_{s=-3} = -7 \]
\[ B = \frac{d}{ds} \left. \left( (s+2)^2 \cdot \frac{2s^2 + 13s + 14}{(s+2)^2 \cdot (s+3)} \right) \right|_{s=-2} = \left. \frac{2s^2 + 12s + 25}{(s+3)^2} \right|_{s=-2} = 9 \]
\[ C = \left. (s+2)^2 \cdot \frac{2s^2 + 13s + 14}{(s+2)^2 \cdot (s+3)} \right|_{s=-2} = \left. \frac{2s^2 + 13s + 14}{s+3} \right|_{s=-2} = -4 \]
\[ v(t) = \mathcal{L}^{-1} [V(s)] = \mathcal{L}^{-1} \left[- \frac{7}{s+3} + \frac{9}{s+2} - \frac{4}{(s+2)^2} \right] \]
La risposta totale del sistema è $v(t) = \left( -7 \cdot e^{-3t} + 9 \cdot e^{-2t} - 4t \cdot e^{-2t} \right) \cdot \delta_{-1}(t)$


\newpage
\section{Diagrammi di flusso}

\subsection{Esercizio 1}
\paragraph{Testo:}
trovare la funzione di trasferimento tra ingresso e uscita del seguente diagramma di flusso
\begin{figure}[htp]
	\centering
	\includegraphics[width=\textwidth, height=\textheight, keepaspectratio]{flusso.png}
	\caption{Diagramma di flusso}
	\label{fig:flusso}
\end{figure}

\paragraph{Cammini aperti:}
\[ P_1 = \frac{R_3 \cdot R_4}{R_1 \cdot R_2} \]
\paragraph{Anelli singoli:}
\[ P_{11} = - \frac{R_3}{R_1} \]
\[ P_{21} = - \frac{R_3}{R_2} \]
\[ P_{31} = - \frac{R_4}{R_2} \]
\paragraph{Coppie di anelli che non si toccano:}
\[ P_{12} = \left( - \frac{R_3}{R_1} \right) \cdot \left( - \frac{R_4}{R_2} \right) = \frac{R_3 \cdot R_4}{R_1 \cdot R_2} \]
\paragraph{Delta:}
\[ \Delta = 1 - \left( - \frac{R_3}{R_1} - \frac{R_3}{R_2} - \frac{R_4}{R_2} \right) + \frac{R_3 \cdot R_4}{R_1 \cdot R_2} \]
\[ \Delta_1 = 1 \]
\paragraph{Trasmittanza totale:}
\[
T = \frac{\frac{R_3 \cdot R_4}{R_1 \cdot R_2} \cdot 1}{\frac{R_1 \cdot R_2 + R_3 \cdot R_2 + R_3 \cdot R_1 + R_4 \cdot R_1 + R_3 \cdot R_4}{R_1 \cdot R_2}} = \]
\[ = \frac{R_3 \cdot R_4}{R_1 \cdot R_2 + R_3 \cdot R_2 + R_3 \cdot R_1 + R_4 \cdot R_1 + R_3 \cdot R_4}
\]


\section{Diagrammi di Bode}

\subsection{Esercizio 1}
\paragraph{Testo:}
tracciare i diagrammi di Bode della seguente funzione di trasferimento
\[ G(s) = \frac{s+1}{s^4 + 2s^3 + 100s^2} = \frac{s+1}{s^2 \cdot (s^2 + 2s + 100)} \]

\paragraph{Forma di Bode:}
\[ G(j\omega) = \frac{1}{100} \cdot \frac{1}{(j\omega)^2} \cdot \frac{1+j\omega}{1 + \frac{2}{100} \cdot j\omega - \frac{\omega^2}{100}} \]

\paragraph{Termine costante:}
\[ H(j\omega)= 10^{-2} \]
\begin{itemize}
	\item Modulo
	\[ |H(j\omega)|_{dB} = 20 \cdot log|10^{-2}| = -40 \, dB \]
	\item Fase
	\[ arg(H(j\omega)) = 0 \]
\end{itemize}
\begin{figure}[htp]
	\centering
	\includegraphics[scale=0.5]{costante.png}
	\caption{Termine costante}
	\label{fig:costante}
\end{figure}
\newpage

\paragraph{Zeri e poli nell'origine:}
\[ H(j\omega)=\frac{1}{(j\omega)^2} \]
\begin{itemize}
	\item Modulo
	\[ |H(j\omega)|_{dB} = 20 \cdot (-2) \cdot log|\omega| = -40 \, \frac{dB}{dec}  \]
	\item Fase
	\[ arg(H(j\omega)) = -2 \cdot 90 = -180 \]
\end{itemize}
\begin{figure}[htp]
	\centering
	\includegraphics[scale=0.5]{origine.png}
	\caption{Polo nell'origine}
	\label{fig:origine}
\end{figure}

\paragraph{Zeri e poli reali:}
\[ \tau = 1, \, \mu = 1 \]
\[ H(j\omega)=1+j\omega \]
\begin{itemize}
	\item Modulo
	\[
	|H(j\omega)|_{dB} =
	\begin{cases}
	0& se \,\, \omega << 10^0 \\
	20 \, \frac{dB}{dec}& se \,\, \omega >> 10^0
	\end{cases}
	\]
	\item Fase
	\[
	arg(H(j\omega)) =
	\begin{cases}
	0& se \,\, \omega << 10^0 \\
	90& se \,\, \omega >> 10^0
	\end{cases}
	\]
	\paragraph{Approssimazione:}
	\[ A = \left( 0.2, 0 \right) \]
	\[ B = \left( 5, 90 \right) \]
\end{itemize}
\newpage
\begin{figure}[htp]
	\centering
	\includegraphics[scale=0.5]{reali.png}
	\caption{Zero reale}
	\label{fig:reali}
\end{figure}

\paragraph{Zeri e poli complessi coniugati:}
\[ \omega_n=\sqrt{100}=10, \, \frac{2}{10}\zeta=\frac{2}{100} \longrightarrow \zeta=\frac{1}{10}, \, \mu=-1 \]
\[ H(j\omega) = \left( 1 + \frac{2}{100} \cdot j\omega - \frac{\omega^2}{100} \right)^{-1} \]
\begin{itemize}
	\item Modulo
	\[
	|H(j\omega)|_{dB} =
	\begin{cases}
	0& se \,\, \omega << 10^1 \\
	-40 \, \frac{dB}{dec}& se \,\, \omega >> 10^1
	\end{cases}
	\]
	\item Fase
	\[
	arg(H(j\omega)) =
	\begin{cases}
	0& se \,\, \omega << 10^1 \\
	-180& se \,\, \omega >> 10^1
	\end{cases}
	\]
	\paragraph{Approssimazione:}
	\[ A = \left( 8.5, 0 \right) \]
	\[ B = \left( 11.7, -180 \right) \]
\end{itemize}
\begin{figure}[htp]
	\centering
	\includegraphics[scale=0.5]{complessi.png}
	\caption{Polo complesso}
	\label{fig:complessi}
\end{figure}
\newpage

\paragraph{Diagramma globale:}
unendo tutti i diagrammi si ricava il seguente
\begin{figure}[htp]
	\centering
	\includegraphics[scale=0.5]{globale.png}
	\caption{Diagramma globale}
	\label{fig:globale}
\end{figure}


\newpage
\section{Sistemi LTI a tempo discreto}

\subsection{Esercizio 1}
\paragraph{Testo:}
\[ v(k) + v(k-1) = u(k) - u(k-1) \]

\paragraph{Stabilità asintotica:}
\[ \left. z^0 + z^{-1} = 0 \right|_{\cdot z^n = z^1} \]
\[ z + 1 = 0 \longrightarrow \lambda_1 = -1 \]
Il sistema non è asintoticamente stabile perché $|\lambda_1|=1$

\paragraph{BIBO stabilità:}
\[ H(z) = \frac{z-1}{z+1} \]
Il sistema non è BIBO stabile perché $|p_1|=1$

\paragraph{Risposta impulsiva (ricorsione):}
\[
\begin{cases}
v(k) = h(k) \\
u(k) = \delta(k)
\end{cases}
\longrightarrow
h(k) + h(k-1) = \delta(k) - \delta(k-1)
\]
\[
k=0
\longrightarrow
h(0) + h(-1) = \delta(0) - \delta(-1)
\longrightarrow
h(0) = 1
\]
\[
k=1
\longrightarrow
h(1) + h(0) = \delta(1) - \delta(0)
\longrightarrow
h(1) = -2
\]
\[
k=2
\longrightarrow
h(2) + h(1) = \delta(2) - \delta(1)
\longrightarrow
h(2) = 2
\]
\[
h(k) =
\begin{cases}
1& se \,\, k = 0 \\
-2& se \,\, k \geq 1 \,\, e \,\, dispari \\
2& se \,\, k \geq 1 \,\, e \,\, pari
\end{cases}
\]
\[ h(k) = d_0 \cdot \delta(k) + d_1 \cdot (-1)^k \cdot \delta_{-1}(k-1) \]
\[
\begin{cases}
h(0) = d_0 \cdot \delta(0) + d_1 \cdot (-1)^0 \cdot \delta_{-1}(-1) \\
h(1) = d_0 \cdot \delta(1) + d_1 \cdot (-1)^1 \cdot \delta_{-1}(0)
\end{cases}
\longrightarrow
\begin{cases}
d_0 = 1 \\
- d_1 = - 2
\end{cases}
\longrightarrow
\begin{cases}
d_0 = 1 \\
d_1 = 2
\end{cases}
\]
La risposta impulsiva del sistema è $h(k) = \delta(k) + 2 \cdot (-1)^k \cdot \delta_{-1}(k-1)$

\paragraph{Risposta forzata (ricorsione):}
\[
k=0
\longrightarrow
v(0) + v(-1) = u(0) - u(-1)
\longrightarrow
v(0) = u(0)
\]
\[
k=1
\longrightarrow
v(1) + v(0) = u(1) - u(0)
\longrightarrow
v(1) = u(1) - 2u(0)
\]
\[
k=2
\longrightarrow
v(2) + v(1) = u(2) - u(1)
\longrightarrow
v(2) = u(2) - 2u(1) + 2u(0)
\]
La risposta forzata del sistema è
\[ v_f(k) = u(k) + \sum_{i=0}^{k-1} 2 \cdot (-1)^{k-i} \cdot u(i) \]

\paragraph{Risposta forzata (convoluzione):}
\[ h(k-i) = \delta(k-i) + 2 \cdot (-1)^{k-i} \cdot \delta_{-1}(k-i-1) \]
\[ v_f(k) = [u*h](k) = \sum_{i=0}^k u(i) \cdot h(k-i) = \]
\[ = \sum_{i=0}^k u(i) \cdot [\delta(k-i) + 2 \cdot (-1)^{k-i} \cdot \delta_{-1}(k-i-1)] = \]
\[ = [\delta(k-k) + 2 \cdot (-1)^{k-k} \cdot \delta_{-1}(k-k-1)] \cdot u(k) + \sum_{i=0}^{k-1} 2 \cdot (-1)^{k-i} \cdot u(i) = \]
\[ = [\delta(0) + 2 \cdot (-1)^{0} \cdot \delta_{-1}(-1)] \cdot u(k) + \sum_{i=0}^{k-1} 2 \cdot (-1)^{k-i} \cdot u(i) = \]
\[ = [1 + 0] \cdot u(k) + \sum_{i=0}^{k-1} 2 \cdot (-1)^{k-i} \cdot u(i) = u(k) + \sum_{i=0}^{k-1} 2 \cdot (-1)^{k-i} \cdot u(i) \]


\subsection{Esercizio 2}
\paragraph{Testo:}
\[ v(k) - \frac{3}{10} v(k-1) - \frac{1}{10} v(k-2) = u(k) - \frac{2}{5} u(k-2) \]
\[ v(-1) = 2, \, v(-2) = -2, \, u(k) = 2^k \cdot \delta_{-1}(k) \]

\paragraph{Stabilità:}
\[ \left. z^0 - \frac{3}{10} z^{-1} - \frac{1}{10} z^{-2} = 0 \right|_{\cdot z^n = z^2} \]
\[ z^2 - \frac{3}{10} z - \frac{1}{10} = 0 \]
\[ \lambda_{1,2} = \frac{\frac{3}{10} \pm \sqrt{\frac{9}{100} + \frac{4}{10}}}{2} = \frac{\frac{3}{10} \pm \frac{7}{10}}{2}\]
\[ \lambda_1 = \frac{1}{2}, \, \lambda_2 = - \frac{1}{5} \]
Il sistema è asintoticamente stabile e quindi anche BIBO stabile

\paragraph{Risposta libera:}
\[ v_l(k) = c_1 \cdot \left( \frac{1}{2} \right)^k + c_2 \cdot \left( - \frac{1}{5} \right)^k \]
\[
\begin{cases}
2c_1 - 5c_2 = 2 \\
4c_1 + 25c_2 = -2
\end{cases}
\longrightarrow
\begin{cases}
c_1 = 1 + \frac{5}{2} c_2 \\
4 + 10c_2 + 25c_2 = -2
\end{cases}
\longrightarrow
\begin{cases}
c_1 = \frac{4}{7} \\
c_2 = - \frac{6}{35}
\end{cases}
\]
La risposta libera del sistema è $v_l(k) = \frac{4}{7} \cdot \left( \frac{1}{2} \right)^k - \frac{6}{35} \cdot \left( - \frac{1}{5} \right)^k$

\paragraph{Risposta impulsiva (fratti semplici):}
\[ H(z) = \frac{z^2 - \frac{2}{5}}{z^2 - \frac{3}{10} z - \frac{1}{10}} = \frac{z^2 - \frac{2}{5}}{(z - \frac{1}{2})(z + \frac{1}{5})} \]
\[ H_1(z) = \frac{H(z)}{z} = \frac{z^2 - \frac{2}{5}}{z(z - \frac{1}{2})(z + \frac{1}{5})} = \frac{A}{z} + \frac{B}{z - \frac{1}{2}} + \frac{C}{z + \frac{1}{5}} \]
\[
A = \left. z \cdot \frac{z^2 - \frac{2}{5}}{z(z - \frac{1}{2})(z + \frac{1}{5})} \right|_{z=0} = \left. \frac{z^2 - \frac{2}{5}}{(z - \frac{1}{2})(z + \frac{1}{5})} \right|_{z=0} = 4
\]
\[
B = \left. \left( z - \frac{1}{2} \right) \cdot \frac{z^2 - \frac{2}{5}}{z(z - \frac{1}{2})(z + \frac{1}{5})} \right|_{z = \frac{1}{2}} = \left. \frac{z^2 - \frac{2}{5}}{z(z + \frac{1}{5})} \right|_{z = \frac{1}{2}} = - \frac{3}{7}
\]
\[
C = \left. \left( z + \frac{1}{5} \right) \cdot \frac{z^2 - \frac{2}{5}}{z(z - \frac{1}{2})(z + \frac{1}{5})} \right|_{z = - \frac{1}{5}} = \left. \frac{z^2 - \frac{2}{5}}{z(z - \frac{1}{2})} \right|_{z = - \frac{1}{5}} = - \frac{18}{7}
\]
\[ H_1(z) = \frac{4}{z} - \frac{3}{7} \cdot \frac{1}{z - \frac{1}{2}} - \frac{18}{7} \cdot \frac{1}{z + \frac{1}{5}} \]
\[ H(z) = H_1(z) \cdot z = 4 - \frac{3}{7} \cdot \frac{z}{z - \frac{1}{2}} - \frac{18}{7} \cdot \frac{z}{z + \frac{1}{5}} \]
\[
h(k) = \mathcal{Z} [H(z)] = 4 \cdot \delta(k) + \left[ - \frac{3}{7} \cdot \left( \frac{1}{2} \right)^k - \frac{18}{7} \cdot \left( - \frac{1}{5} \right)^k \right] \cdot \delta_{-1}(k)
\]

\paragraph{Risposta forzata (fratti semplici):}
\[ U(z) = \mathcal{Z} \left[ 2^k \cdot \delta_{-1}(k) \right] = \frac{z}{z-2} \]
\[ V_f(z) = H(z) \cdot U(z) = \frac{z^2 - \frac{2}{5}}{(z - \frac{1}{2})(z + \frac{1}{5})} \cdot \frac{z}{z-2} \]
\[ V_{f_1}(z) = \frac{V_f(z)}{z} = \frac{z^2 - \frac{2}{5}}{(z - \frac{1}{2})(z + \frac{1}{5})(z-2)} = \frac{A}{z - \frac{1}{2}} + \frac{B}{z + \frac{1}{5}} + \frac{C}{z-2} \]
\[
A = \left. \left( z - \frac{1}{2} \right) \cdot \frac{z^2 - \frac{2}{5}}{(z - \frac{1}{2})(z + \frac{1}{5})(z-2)} \right|_{z = \frac{1}{2}} = \left. \frac{z^2 - \frac{2}{5}}{(z + \frac{1}{5})(z-2)} \right|_{z = \frac{1}{2}} = \frac{1}{7}
\]
\[
B = \left. \left( z + \frac{1}{5} \right) \cdot \frac{z^2 - \frac{2}{5}}{(z - \frac{1}{2})(z + \frac{1}{5})(z-2)} \right|_{z = - \frac{1}{5}} = \left. \frac{z^2 - \frac{2}{5}}{(z - \frac{1}{2})(z-2)} \right|_{z = - \frac{1}{5}} = - \frac{18}{77}
\]
\[
C = \left. \left( z - 2 \right) \cdot \frac{z^2 - \frac{2}{5}}{(z - \frac{1}{2})(z + \frac{1}{5})(z-2)} \right|_{z = 2} = \left. \frac{z^2 - \frac{2}{5}}{(z - \frac{1}{2})(z + \frac{1}{5})} \right|_{z = 2} = \frac{12}{11}
\]
\[ V_{f_1}(z) = \frac{1}{7} \cdot \frac{1}{z - \frac{1}{2}} - \frac{18}{77} \cdot \frac{1}{z + \frac{1}{5}} + \frac{12}{11} \cdot \frac{1}{z-2} \]
\[ V_f(z) = V_{f_1}(z) \cdot z = \frac{1}{7} \cdot \frac{z}{z - \frac{1}{2}} - \frac{18}{77} \cdot \frac{z}{z + \frac{1}{5}} + \frac{12}{11} \cdot \frac{z}{z-2} \]
\[
v_f(k) = \mathcal{Z} [V_f(z)] = \left[ \frac{1}{7} \cdot \left( \frac{1}{2} \right)^k - \frac{18}{77} \cdot \left( - \frac{1}{5} \right)^k + \frac{12}{11} \cdot \left( 2 \right)^k \right] \cdot \delta_{-1}(k)
\]


\subsection{Esercizio 3}
\paragraph{Testo:}
\[ v(k) - v(k-1) + \frac{1}{4} v(k-2) = u(k) - 3 u(k-1) \]
\[ v(-1) = 4, \, v(-2) = 3, \, u(k) = \left( -\frac{1}{2} \right)^k \cdot \delta_{-1}(k) \]

\paragraph{Stabilità:}
\[ \left. z^0 - z^{-1} + \frac{1}{4} z^{-2} = 0 \right|_{\cdot z^n = z^2} \]
\[ z^2 - z + \frac{1}{4} = 0 \]
\[ \lambda_{1,2} = \frac{1 \pm \sqrt{1-1}}{2} = \frac{1 \pm 0}{2} = \frac{1}{2} \]
Il sistema è asintoticamente stabile e quindi anche BIBO stabile

\paragraph{Risposta libera:}
\[ v_l(k) = c_1 \cdot \left( \frac{1}{2} \right)^k + c_2 \cdot k \cdot \left( \frac{1}{2} \right)^k \]
\[
\begin{cases}
2c_1 - 2c_2 = 4 \\
4c_1 - 8c_2 = 3
\end{cases}
\longrightarrow
\begin{cases}
c_1 = 2 + c_2 \\
8 + 4c_2 - 8c_2 = 3
\end{cases}
\longrightarrow
\begin{cases}
c_1 = \frac{13}{4} \\
c_2 = \frac{5}{4}
\end{cases}
\]
La risposta libera del sistema è $v_l(k) = \frac{13}{4} \cdot \left( \frac{1}{2} \right)^k + \frac{5}{4} \cdot k \cdot \left( \frac{1}{2} \right)^k$

\paragraph{Risposta totale:}
\[ U(z) = \mathcal{Z} \left[ \left( -\frac{1}{2} \right)^k \cdot \delta_{-1}(k) \right] = \frac{z}{z+\frac{1}{2}} \]
\[ z^0 \cdot V(z) - (z^{-1} \cdot V(z) + 4 z^0) + \frac{1}{4} \cdot (z^{-2} \cdot V(z) + 4 z^{-1} + 3 z^0) = \]
\[ \left. = z^0 \cdot U(z) - 3 z^{-1} \cdot U(z) \right|_{\cdot z^n = z^2} \]
\[ V(z) \cdot \left( z^2 - z + \frac{1}{4} \right) -4z^2 + z + \frac{3}{4} z^2 = U(z) \cdot (z^2 - 3z) \]
\[ V(z) = \frac{z^2 -3z}{z^2 - z + \frac{1}{4}} \cdot U(z) + \frac{\frac{13}{4} z^2 - z}{z^2 - z + \frac{1}{4}} = \frac{z^3 -3z^2}{(z-\frac{1}{2})^2 \cdot (z+\frac{1}{2})} + \frac{\frac{13}{4} z^2 - z}{(z-\frac{1}{2})^2} = \]
\[ = \frac{z^3 -3z^2 + \frac{13}{4} z^3 - z^2 + \frac{13}{8} z^2 - \frac{1}{2} z}{(z-\frac{1}{2})^2 \cdot (z+\frac{1}{2})} = \frac{\frac{17}{4} z^3 - \frac{19}{8} z^2 - \frac{1}{2} z}{(z-\frac{1}{2})^2 \cdot (z+\frac{1}{2})} \]
\[
V_1(z) = \frac{V(z)}{z} = \frac{\frac{17}{4} z^2 - \frac{19}{8} z - \frac{1}{2}}{(z-\frac{1}{2})^2 \cdot (z+\frac{1}{2})}
= \frac{A}{z+\frac{1}{2}} + \frac{B}{z-\frac{1}{2}} + \frac{C}{(z-\frac{1}{2})^2}
\]
\newpage
\[
A = \left. \left( z+\frac{1}{2} \right) \cdot \frac{\frac{17}{4} z^2 - \frac{19}{8} z - \frac{1}{2}}{(z-\frac{1}{2})^2 \cdot (z+\frac{1}{2})} \right|_{z=-\frac{1}{2}} = \left. \frac{\frac{17}{4} z^2 - \frac{19}{8} z - \frac{1}{2}}{(z-\frac{1}{2})^2} \right|_{z=-\frac{1}{2}} = \frac{7}{4}
\]
\[
B = \frac{d}{dz} \left. \left( \left( z-\frac{1}{2} \right)^2 \cdot \frac{\frac{17}{4} z^2 - \frac{19}{8} z - \frac{1}{2}}{(z-\frac{1}{2})^2 \cdot (z+\frac{1}{2})} \right) \right|_{z=\frac{1}{2}} = \left. \frac{\frac{17}{4} z^2 + \frac{17}{4} z - \frac{11}{16}}{(z+\frac{1}{2})^2} \right|_{z=\frac{1}{2}} = \frac{5}{2}
\]
\[
C = \left. \left( z-\frac{1}{2} \right)^2 \cdot \frac{\frac{17}{4} z^2 - \frac{19}{8} z - \frac{1}{2}}{(z-\frac{1}{2})^2 \cdot (z+\frac{1}{2})} \right|_{z=\frac{1}{2}} = \left. \frac{\frac{17}{4} z^2 - \frac{19}{8} z - \frac{1}{2}}{z+\frac{1}{2}} \right|_{z=\frac{1}{2}} = -\frac{5}{8}
\]
\[ V_1(z) = \frac{7}{4} \cdot \frac{1}{z+\frac{1}{2}} + \frac{5}{2} \cdot \frac{1}{z-\frac{1}{2}} - \frac{5}{8} \cdot \frac{1}{(z-\frac{1}{2})^2} \]
\[
V(z) = V_1(z) \cdot z = \frac{7}{4} \cdot \frac{z}{z+\frac{1}{2}} + \frac{5}{2} \cdot \frac{z}{z-\frac{1}{2}} - \frac{5}{8} \cdot \frac{z}{(z-\frac{1}{2})^2} =
\]
\[
= \frac{7}{4} \cdot \frac{z}{z+\frac{1}{2}} + \frac{5}{2} \cdot \frac{z}{z-\frac{1}{2}} - \frac{5}{4} \cdot \frac{z \cdot \frac{1}{2}}{(z-\frac{1}{2})^2}
\]
\[
v(k) = \mathcal{Z} [V(z)] = \left[ \frac{7}{4} \cdot \left( -\frac{1}{2} \right)^k + \frac{5}{2} \cdot \left( \frac{1}{2} \right)^k - \frac{5}{4} \cdot k \cdot \left( \frac{1}{2} \right)^k \right] \cdot \delta_{-1}(k)
\]


\backmatter
\chapter{Credits}
Repository github: \url{https://github.com/zampierida98/UniVR-informatica} \\
Indirizzo e-mail: \mail{zampieri.davide@outlook.com}


\end{document}